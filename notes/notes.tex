\documentclass[a4paper, 12pt]{report}
\usepackage{graphicx}
\usepackage{amsmath, amsthm, amssymb}
\usepackage{enumerate}
\usepackage{enumitem}
\usepackage{array}
\usepackage{tabularx}
\usepackage{hyperref}
\usepackage{caption}
\usepackage[normalem]{ulem}
\usepackage{pdfpages}
\usepackage[toc,page]{appendix}
\usepackage{minted}

%%% blank footnote
\newcommand\blfootnote[1]{
	\begingroup
	\renewcommand\thefootnote{}\footnote{#1}
	\addtocounter{footnote}{-1}
	\endgroup
}
%%%

%%% prevent hyphenation
\tolerance=1
\emergencystretch=\maxdimen
\hyphenpenalty=10000
\hbadness=10000
%%%

%%% make inline math look like display style math
\everymath{\displaystyle}

%%% minimize
\DeclareMathOperator*{\minimize}{minimize}
%%%

%%%
\def\N{\mathbb{N}}
\def\Z{\mathbb{Z}}
\def\Q{\mathbb{Q}}
\def\R{\mathbb{R}}
\def\F{\mathbb{F}}

\def\tends{\rightarrow}
\def\into{\rightarrow}
\def\half{\frac{1}{2}}
\def\quarter{\frac{1}{4}}

\newcommand{\set}[1]{\left\{ #1 \right\}}
\newcommand{\norm}[1]{\left\Vert #1 \right\Vert}
\newcommand{\card}[1]{\left\vert #1 \right\vert}
\newcommand{\ceil}[1]{\left\lceil #1 \right\rceil}
\newcommand{\floor}[1]{\left\lfloor #1 \right\rfloor}

\DeclareMathOperator{\dia}{dia}
%%%

\title{Notes\\
\large Machine Learning by Andrew Ng on Coursera}
\author{Sparsh Jain}
\date{\today}

\begin{document}

\maketitle
\tableofcontents
% \listoffigures
% \listoftables

\chapter{Introduction}
\emph{Machine learning} (task, experience, performance) can be classified into
\emph{Supervised} and \emph{Unsupervised} learning.

\section{Supervised Learning}
Supervised learning can be basically classified into \emph{Regression} and
\emph{Classification} problems.

\subsection{Regression Problem}
Regression problems work loosely on continuous range of outputs.

\subsection{Classification Problems}
Classification problems work loosely on discrete range of outputs.

\section{Unsupervised Learning}
An example is \emph{Clustering Problem}.

% \blfootnote{Check \autoref{lecture1} for more details.}
\blfootnote{Check \href{lecture_pdf/Lecture1.pdf}{Lecture1.pdf} for more details.}

\chapter{Linear Regression with One Variable}

\section{Notations}
\begin{align*}
    m                  & = \text{number of training examples}             \\
    x\text{'s}         & = \text{`input' variables / features}            \\
    y\text{'s}         & = \text{`output' variables / `target' variables} \\
    (x, y)             & = \text{single training example}                 \\
    (x^{(i)}, y^{(i)}) & = i^{th} \  \text{example}                       \\
\end{align*}

\section{Supervised Learning}
We have a data set (\emph{Training Set}).

Training Set $\rightarrow$ Learning Algorithm $\rightarrow$ $h$ (\emph{hypothesis},
a function $X \to Y$)

\subsection*{To Represent \texorpdfstring{$h$}{}}
\begin{equation*}
    h_\theta(x) = \theta_0 + \theta_1x
\end{equation*}

\subsection*{Cost}
\begin{equation*}
    \minimize_{\theta_0,\ \theta_1} \frac{1}{2m}\sum_1^m(h_\theta(x) - y)^2
\end{equation*}
\subsubsection*{Cost Function}
Squared Error Cost Function
\begin{equation*}
    J(\theta_0, \theta_1) = \frac{1}{2m}\sum_1^m(h_\theta(x) - y)^2
\end{equation*}
\subsection*{}
\begin{equation*}
    \minimize_{\theta_0,\ \theta_1} J(\theta_0, \theta_1)
\end{equation*}

\section{Gradient Descent}
Finds local optimum:
\begin{enumerate}
    \item Start with some value
    \item Get closer to optimum
\end{enumerate}

\subsection*{Algorithm}
\begin{equation*}
    \theta_j := \theta_j - \alpha\frac{\partial}{\partial\theta_j}J(\theta) \ \forall j
\end{equation*}

where $\alpha$ = learning rate

\subsubsection*{Important!}
Simultaneous Update!
\begin{align*}
    temp_j   & := \theta_j - \alpha\frac{\partial}{\partial\theta_j}J(\theta) \ \forall j \\
    \theta_j & := temp_j \ \forall j
\end{align*}

\section{Gradient Descent for Linear Regression}
Cost function for linear regression is convex!

\emph{Batch Gradient Descent}: Each step of gradient descent uses all training examples.

% \blfootnote{Check \autoref{lecture2} for more details.}
\blfootnote{Check \href{lecture_pdf/Lecture2.pdf}{Lecture2.pdf} for more details.}

\chapter{Linear Algebra}
\section{Matrix}
Rectangular array of numbers:
$$
    \begin{bmatrix}
        1 & 2 & 3 \\
        4 & 5 & 6 \\
    \end{bmatrix}
$$

\paragraph*{Dimension of the matrix:} \#rows x \#cols (2 x 3)

\paragraph*{Elements of the matrix:}
\begin{align*}
    A      & =	\begin{bmatrix}
        1 & 2 & 3 \\
        4 & 5 & 6 \\
    \end{bmatrix}                                 \\
    A_{ij} & = \text{``$i,j$ entry'' in the $i^{th}$ row, $j^{th}$ col} \\
\end{align*}

\section{Vector}
An $n \times 1$ matrix.
\begin{align*}
    y & =	\begin{bmatrix}
        1 \\
        2 \\
        3 \\
        4 \\
    \end{bmatrix} \\
    y_i = i^{th} \text{ element}   \\
\end{align*}

\paragraph*{Note:} Uppercase for matrices, lowercase for vectors.

\section{Addition and Scalar Multiplication}
Add/Subtract (element by element) matrices of same dimention only!

Multiply/Divide (all elements) a matrix by scalar!

\section{Matrix Matrix Multiplication}
$m \times n$ matrix multiplied by $n \times o$ matrix gives a $m \times o$ matrix.

\subsection*{Properties}
\begin{enumerate}
    \item Matrix Multiplication is \emph{not} Commutative.
    \item Matrix Multiplication is Associative.
    \item \emph{Identity Matrix ($I$):} $1$'s along diagonal, $0$'s everywhere else in an
          $n \times n$ matrix. $AI = IA = A$.
\end{enumerate}

\section{Inverse and Transpose}

\subsection*{Inverse}
Only square ($n \times n$) matrices \emph{may} have an inverse.

$$AA^{-1} = A^{-1}A = I$$.

Matrices that don't have an inverse	are \emph{singular} or \emph{degenerate} matrices.

\subsection*{Transpose}
Let $A$ be an $m \times n$ matrix and let $B = A^T$, then
$$ B_{ij} = A_{ji} $$
Example:
\begin{align*}
    A       & =	\begin{bmatrix}
        1 & 2 & 3 \\
        4 & 5 & 6 \\
    \end{bmatrix} \\
    B = A^T & =	\begin{bmatrix}
        1 & 4 \\
        2 & 5 \\
        3 & 6 \\
    \end{bmatrix} \\
\end{align*}

% \blfootnote{Check \autoref{lecture3} for more details.}
\blfootnote{Check \href{lecture_pdf/Lecture3.pdf}{Lecture3.pdf} for more details.}


\chapter{Linear Regression with Multiple Variables}

\section{Notations}
\begin{align*}
    n         & = \text{number of features}                                \\
    x^{(i)}   & = \text{input (features) of $i^{th}$ training example}     \\
    x_j^{(i)} & = \text{value of feature $j$ of $i^{th}$ training example} \\
\end{align*}

\section{Hypothesis}
\subsection*{Previously:}
\begin{equation*}
    h_\theta(x) = \theta_0 + \theta_1x\\
\end{equation*}
\subsection*{Now:}
\begin{equation*}
    h_\theta(x) = \theta_0	+ \theta_1x_1 + \theta_2x_2 + \dots + \theta_nx_n\\
\end{equation*}

For convinience, define $x_0 = 1$. So
\begin{align*}
    h_\theta(x) & = \theta_0x_0	+ \theta_1x_1 + \theta_2x_2 + \dots + \theta_nx_n   \\
    \\
    x           & = \begin{bmatrix}
        x_0    \\
        x_1    \\
        x_2    \\
        \vdots \\
        x_n    \\
    \end{bmatrix} \in \R^{n+1}                       &
    \theta      & = \begin{bmatrix}
        \theta_0 \\
        \theta_1 \\
        \theta_2 \\
        \vdots   \\
        \theta_n \\
    \end{bmatrix} \in \R^{n+1}                         \\
    \\
    h_\theta(x) & = \theta^Tx                                                      \\
\end{align*}

\section{Gradient Descent}

\begin{align*}
    \text{Hypothesis}       & :  h_\theta(x) = \theta^Tx
                            & = \theta_0x_0 + \theta_1x_1 + \dots + \theta_nx_n              \\
    \text{Parameters}       & : \theta
                            & =\theta_0, \theta_1, \dots, \theta_n                           \\
    \text{Cost Function}    & : J(\theta) = J(\theta_0, \theta_1, \dots, \theta_n)
                            & =\frac{1}{2m}\sum_{i=1}^m(h_\theta(x^{(i)}) - y^{(i)})^2       \\
    \text{Gradient Descent} & :                                                              \\
                            & \text{Repeat} \{                                               \\
    \theta_j                & : =
    \theta_j - \alpha\frac{\partial}{\partial\theta_j}J(\theta)                              \\
                            & =
    \theta_j - \alpha\frac{\partial}{\partial\theta_j}J(\theta_0, \theta_1, \dots, \theta_n) \\
                            & =
    \theta_j - \alpha\frac{1}{m}\sum_{i=1}^m(h_\theta(x^{(i)} - y^{(i)})x_j^{(i)})           \\
    \}                      & (\text{simultaneously update }
    \forall j = 0, 1, \dots, n)
\end{align*}

\subsection{Feature Scaling}
\paragraph{Idea:} Make sure features are on a similar scale.

Get every feature into approximately a $-1 \le x_i \le 1$ range.

\subsection{Mean Normalization}
Replace $x_i$ with $x_i - \mu_i$ to make features have approximately zero mean
(Do not apply to $x_0 = 1$).

\subsection*{General Rule}
\begin{equation*}
    x_i \leftarrow \frac{x_i - \mu_i}{S_i}
\end{equation*}
where
\begin{align*}
    \mu_i & = \text{average value of } x_i            \\
    S_i   & = \text{range (max - min)}           & or \\
          & = \sigma \text{(standard deviation)}
\end{align*}

\subsection{Learning Rate}
$J(\theta)$ should decrease after every iteration. $\#$iterations vary a lot.

Example \emph{Automatic Convergence Test:} Declare convergence if $J(\theta)$
decreases by less than $\epsilon\ (\text{say } 10^{-3})$ in one iteration.

If $J(\theta)$ increases, use smaller $\alpha$. Too small $\alpha$ means slow convergence.

To choose $\alpha$, try $\dots,\ 0.001,\ 0.003,\ 0.01,\ 0.03,\ 0.1,\ 0.3,\ 1,\ \dots$

\section{Features and Polynomial Regression}
\subsection{Features}
Get an insight in your problem and choose better features (may even combine/separate features).

Ex: size = length $\into$ breadth.

\subsection{Polynomial Regression}
Ex: \begin{align*}
    x_1 & = size   \\
    x_2 & = size^2 \\
    x_3 & = size^3 \\
\end{align*}

\section{Normal Equation}
Solve for $\theta$ analytically!
\begin{align*}
    x^{(i)} & = \begin{bmatrix}
        x_0^{(i)} \\
        x_1^{(i)} \\
        \vdots    \\
        x_n^{(i)} \\
    \end{bmatrix} & \in & \ \R^{n+1}            \\
    X       & = \begin{bmatrix}
        (x^{(1)})^T \\
        (x^{(2)})^T \\
        \vdots      \\
        (x^{(m)})^T \\
    \end{bmatrix} & \in & \ \R^{m \times (n+1)} \\
            & = \begin{bmatrix}
        x_0 & x_1 & \dots & x_n \\
    \end{bmatrix} & \in & \ \R^{m \times (n+1)} \\
    y       & = \begin{bmatrix}
        y^{(1)} \\
        y^{(2)} \\
        \vdots  \\
        y^{(m)} \\
    \end{bmatrix} & \in & \ \R^m                \\
    \theta  & = (X^TX)^{-1}X^Ty                                          \\
\end{align*}

Inverse of a matrix grows as $O(n^3)$, use wisely.

\subsection{Non Invertibility of \texorpdfstring{$X^TX$}{}}
Use 'pinv' function in Octave (pseudo-inverse) instead of 'inv' function (inverse).

If $X^TX$ is non-invertible, common causes are
\begin{enumerate}
    \item Redundant features (linearly dependent)
    \item Too many features ($m \le n$). In this case, delete some features or use \emph{regularization}
\end{enumerate}

\blfootnote{Check \href{lecture_pdf/Lecture4.pdf}{Lecture4.pdf} for more details.}

\chapter{Octave Tutorial}
Check \href{lecture_pdf/Lecture5.pdf}{Lecture5.pdf} for more details.


\chapter{Classification}
Classify into categories (binary or multiple).

\section{Logistic Regression}
\begin{align*}
    0 \le\      & h_\theta(x) \le 1                                               \\
    h_\theta(x) & = g(\theta^Tx)                                                  \\
    g(z)        & = \frac{1}{1 + e^{-z}}        &
    \text{$g$ is called a \emph{sigmoid} function or a \emph{logistic} function.} \\
    h_\theta(x) & = \frac{1}{1 + e^{\theta^Tx}}                                   \\
\end{align*}

\subsection*{Interpretation of Hypothesis Output}
\begin{align*}
    h_\theta(x) & = \text{estimated probability that $y = 1$ on input $x$} \\
    h_\theta(x) & = P(y = 1 | x; \theta) =
    \text{probability that $y = 1$, given $x$, parameterized by $\theta$}  \\
\end{align*}
\begin{equation*}
    P(y = 0 | x; \theta) + P(y = 1 | x; \theta) = 1                  \\
\end{equation*}

\section{Decision Boundary}
\begin{align*}
    \text{Predict: } y & = 1 & \text{ if } h_\theta(x) \ge 0.5         \\
                       &     & (\theta^Tx \ge 0)                       \\
    \text{Predict: } y & = 0 & \text{ if } h_\theta(x) < 0.5           \\
                       &     & (\theta^Tx < 0)                         \\
    \theta^Tx          & = 0 & \text{is the \emph{decision boundary.}} \\
\end{align*}

\subsection*{Non-linear Decision Boundaries}
Use same technique as polynomial regression for features.

\section{Cost Function}
\begin{align*}
    \text{Training Set} & :
    \{(x^{(1)}, y^{(1)}), (x^{(2)}, y^{(2)}), ./dots, (x^{(m)}, y^{(m)})\} \\
    x                   & = \begin{bmatrix}
        x_0    \\
        x_1    \\
        \vdots \\
        x_n    \\
    \end{bmatrix} \in \R^{n+1}         \\
    x_0                 & = 1                                              \\
    y                   & \in \{0, 1\}                                     \\
    h_\theta(x)         & = \frac{1}{1 + e^{-\theta^Tx}}                   \\
\end{align*}

\subsection*{How to choose parameter $\theta$?}

\paragraph{Linear Regression:}
\begin{align*}
    J(\theta)                     & =
    \frac{1}{m}\sum_{i = 1}^m\half(h_\theta(x^{(i)}) - y^{(i)})^2      \\
    J(\theta)                     & =
    \frac{1}{m}\sum_{i = 1}^m\mathrm{Cost}(h_\theta(x^{(i)}), y^{(i)}) \\
    \mathrm{Cost}(h_\theta(x), y) & = \half(h_\theta(x) - y)^2         \\
\end{align*}

\paragraph{Logistic Regression:}
\begin{align*}
    \mathrm{Cost}(h_\theta(x), y) & = \begin{cases}
        -\log(h_\theta(x))     & y = 1 \\
        -\log(1 - h_\theta(x)) & y = 0 \\
    \end{cases}     \\
    \mathrm{Cost}(h_\theta(x), y) & = 0 \text{ if } h_\theta(x) = y \\
    \mathrm{Cost}(h_\theta(x), y) &
    \tends \inf \text{ if } y = 0 \text{ and } h_\theta(x) \tends 1 \\
    \mathrm{Cost}(h_\theta(x), y) &
    \tends \inf \text{ if } y = 1 \text{ and } h_\theta(x) \tends 0 \\
\end{align*}

\paragraph{Note:} $y = 0$ or $1$ always.
\begin{align*}
    \mathrm{Cost}(h_\theta(x), y) &
    = -y\log(h_\theta(x)) - (1-y)\log(1 - h_\theta(x))                              \\
    J(\theta)                     &
    = \frac{1}{m}\sum_{i = 1}^m\mathrm{Cost}\left(h_\theta(x^{(i)}), y^{(i)}\right) \\
                                  &
    = -\frac{1}{m}\left[
        \sum_{i=1}^m\left(
        y^{(i)}\log(h_\theta(x^{(i)})) + (1 - y^{(i)})\log(1 - h_\theta(x^{(i)}))
        \right)
        \right]                                                                     \\
\end{align*}

\paragraph{To fit parameters $\theta$:}
\begin{equation*}
    \minimize_\theta J(\theta)
\end{equation*}

\paragraph{To make a prediction given a new $x$:}
\begin{equation*}
    \text{Output } h_\theta(x) = \frac{1}{1 + e^{-\theta^Tx}}
\end{equation*}

\subsubsection{Gradient Descent:}
Simultaneously update all $\theta_j$
\begin{equation*}
    \theta_j := \theta_j - \alpha\frac{\partial}{\partial\theta_j}J(\theta)
\end{equation*}
Plug in the derivative
\begin{equation*}
    \theta_j := \theta_j - \alpha\frac{1}{m}\sum_{i=1}^m\left(
    h_\theta(x^{(i)}) - y^{(i)}
    \right)x_j^{(i)}
\end{equation*}
Don't forget feature scaling!

\section{Advanced Optimization:}
Something better than gradient descent:
\begin{enumerate}
    \item Conjugate Gradient
    \item BFGS
    \item L-BFGS
\end{enumerate}
Advantages:
\begin{enumerate}
    \item No need to manually pick $\alpha$
    \item Often faster than gradient descent
\end{enumerate}
Disadvantages:
\begin{enumerate}
    \item More complex
\end{enumerate}
Use libraries! Beware of bad implementations!

\paragraph{How to use:}
We first need to provide a function that evaluates the following two functions
for a given input value of $\theta$.
\begin{enumerate}
    \item $J(\theta)$
    \item $\frac{\partial}{\partial\theta_j}J(\theta)$
\end{enumerate}

\begin{minted}{octave}
% We can write a single function that can return both of these:
function [jVal, gradient] = costFunction(theta)
	jVal = [...code to compute J(theta)...];
	gradient = [...code to compute derivative of J(theta)...];
end
\end{minted}

Then we can use octave's \emph{fminunc()} optimization algorithm along with the
\emph{optimset()} function that creates an object containing the options we
want to send to \emph{fminunc()}.
\begin{minted}{octave}
options = optimset('GradObj', 'on', 'MaxIter', 100);
initialTheta = zeros(2,1);
	[optTheta, functionVal, exitFlag] = fminunc(@costFunction, 
		initialTheta, options);
\end{minted}

\section{Multi-Class Classification}
\subsection{One vs All}
Build a separate binary classifier $h_\theta^{(i)}(x)$ for each class against all
other classes.
\begin{equation*}
    h\theta^{(i)} = P(y = i | x; \theta)\ \forall i
\end{equation*}
On a new input $x$, to make a prediction, pick the class $i$ that maximizes
$h_\theta^{(i)}(x)$

\blfootnote{Check \href{lecture_pdf/Lecture6.pdf}{Lecture6.pdf} for more details.}

\chapter{Regularization}
\section{Problem of Overfitting}
\begin{enumerate}
    \item Underfitting (High Bias)
    \item Right Fit
    \item Overfitting (High Variance)
\end{enumerate}
\paragraph{Overfitting:} If we have too many features, the learned hypothesis may fit
the training set very well
$\left(J(\theta) = \frac{1}{2m}\sum_{i=1}^m(h\theta(x^{(i)}) - y^{(i)})^2 \approx 0\right)$,
but fail to generalize to new examples (predict prices on new examples).

\section{Addressing Overfitting}
Options:
\begin{enumerate}
    \item Reduce the number of features
          \begin{itemize}
              \item Manually select which features to keep
              \item Model selection algorithm (later)
          \end{itemize}
    \item Regularization
          \begin{itemize}
              \item Keep all the features, but reduce the magnitude/values of parameters
                    $\theta_j$
              \item Works well when we have a lot of features, each of which contributes
                    a bit to predicting $y$
          \end{itemize}
\end{enumerate}

\section{Cost Function}
Say our overfitting hypothesis is
$h_\theta(x) = \theta_0 + \theta_1x + \theta_2x^2 + \theta_3x^3 + \theta_4x^4$.
Suppose, we penalize and make $\theta_3, \theta_4$ very small
\begin{equation*}
    \minimize_\theta\left(
    \frac{1}{2m}\sum_{i=1}^m\left(
        h_\theta(x^{(i)}) - y^{(i)}
        \right)^2 + 1000\theta_3^2 + 1000\theta_4^2
    \right)
\end{equation*}

\subsection*{Regularization}
Small values for parameters $\theta_0, \theta_1, \dots, \theta_n$.
\begin{itemize}
    \item \emph{Simpler} hypothesis
    \item Less prone to overfitting
\end{itemize}

\paragraph{Which parameters to penalize?}
\begin{itemize}
    \item Features: $x_1, x_2, \dots, x_{100}$
    \item Parameters: $\theta_0, \theta_1, \theta_2, \dots, \theta_{100}$
\end{itemize}
\begin{equation*}
    J(\theta) = \frac{1}{2m}\left[
        \sum_{i=1}^m\left(
        h_\theta(x^{(i)}) - y^{(i)}
        \right)^2 + \lambda\sum_{j=1}^n\theta_j^2
        \right]
\end{equation*}
\paragraph{Note:} The convention, we don't regularize $\theta_0$ but it doesn't
make very much difference.

$\lambda$ here is \emph{regularization parameter}.

What if $\lambda$ is set too high (say $10^{10}$)? Underfitting!

\section{Regularized Linear Regression}
\paragraph{Updated $J(\theta)$:}
\begin{align*}
    J(\theta) & = \frac{1}{2m}\left[
        \sum_{i=1}^m\left(
        h_\theta(x^{(i)}) - y^{(i)}
        \right)^2 + \lambda\sum_{j=1}^n\theta_j^2
        \right]                      \\
    \minimize_\theta J(\theta)       \\
\end{align*}

\paragraph{Gradient Descent:}
\begin{align*}
    \text{Repeat}   & \{                                                             \\
    \theta_0        & :=
    \theta_0 - \alpha \frac{1}{m} \sum_{i=1}^m(h_\theta(x^{(i)}) - y^{(i)})x_0^{(i)} \\
    \theta_j        & :=
    \theta_j - \alpha \left[
    \frac{1}{m} \sum_{i=1}^m\left(
    (h_\theta(x^{(i)}) - y^{(i)})x_j^{(i)}
    \right) + \frac{\lambda}{m}\theta_j
    \right]         & \forall j \in \{1, 2, \dots, n\}                               \\
    \equiv \theta_j & :=
    \theta_j(1 - \alpha\frac{\lambda}{m}) - \alpha\frac{1}{m}\sum_{i=1}^m(
    h_\theta(x^{(i)}) - y^{(i)}
    )x_j^{(i)}                                                                       \\
    \}
\end{align*}

\paragraph{Normal Equation:}
\begin{align*}
    X                   & = \begin{bmatrix}
        \left(x^{(1)}\right)^T \\
        \left(x^{(2)}\right)^T \\
        \vdots                 \\
        \left(x^{(m)}\right)^T \\
    \end{bmatrix} & \in        & \ \R^{m\times(n+1)} \\
    y                   & = \begin{bmatrix}
        y^{(1)} \\
        y^{(2)} \\
        \vdots  \\
        y^{(m)} \\
    \end{bmatrix} & \in        & \ \R^m              \\
    \theta              & = \left(
    X^TX + \lambda \begin{bmatrix}
        0 & 0 & 0 & \dots  & 0 \\
        0 & 1 & 0 & \dots  & 0 \\
        0 & 0 & 1 & \dots  & 0 \\
        0 & 0 & 0 & \ddots & 0 \\
        0 & 0 & 0 & 0      & 1 \\
    \end{bmatrix} \left(\in \R^{(n+1)\times(n+1)}\right)
    \right)^{-1} - X^Ty & \in                          & \ \R^{n+1}                       \\
\end{align*}

\subsubsection*{Non-Invertibility}
Suppose $m \le n$,
$$\theta = \left(X^TX\right)^{-1}X^y$$

If $\lambda > 0$,
$$
    \theta = \left(
    X^TX + \lambda \begin{bmatrix}
        0 &   &   &        &   \\
          & 1 &   &        &   \\
          &   & 1 &        &   \\
          &   &   & \ddots &   \\
          &   &   &        & 1 \\
    \end{bmatrix}
    \right)^{-1} - X^Ty
$$
Not a problem!

\section{Regularized Logistic Regression}
\paragraph{Cost Function:}
\begin{equation*}
    J(\theta) = -\left[
        \frac{1}{m}\sum_{i=1}^m\left(
        y^{(i)}\log h_\theta(x^{(i)}) +
        (1 - y^{(i)})\log(1 - h_\theta(x^{(i)})
        \right)
        \right] + \frac{\lambda}{2m}\sum_{j=1}^n\theta_j^2
\end{equation*}

\paragraph{Gradient Descent:}
\begin{align*}
    \text{Repeat}   & \{                                                             \\
    \theta_0        & :=
    \theta_0 - \alpha \frac{1}{m} \sum_{i=1}^m(h_\theta(x^{(i)}) - y^{(i)})x_0^{(i)} \\
    \theta_j        & :=
    \theta_j - \alpha \left[
    \frac{1}{m} \sum_{i=1}^m\left(
    (h_\theta(x^{(i)}) - y^{(i)})x_j^{(i)}
    \right) + \frac{\lambda}{m}\theta_j
    \right]         & \forall j \in \{1, 2, \dots, n\}                               \\
    \equiv \theta_j & :=
    \theta_j(1 - \alpha\frac{\lambda}{m}) - \alpha\frac{1}{m}\sum_{i=1}^m(
    h_\theta(x^{(i)}) - y^{(i)}
    )x_j^{(i)}                                                                       \\
    \}
\end{align*}

\paragraph{Advanced Optimization:}
\begin{minted}{octave}
function [jVal, gradient] = costFunction(theta)
	jVal = [...code to compute J(theta)...];
	gradient = [...code to compute derivative of J(theta)...];
\end{minted}

\blfootnote{Check \href{lecture_pdf/Lecture7.pdf}{Lecture7.pdf} for more details.}

\chapter{Neural Networks}
\section{Non-Linear Hypothesis}
If $\#$features is high, hypothesis could have extremely high
$\#$terms. Hence the need for non-linear hypothesis.

\section{Neural Networks}
\begin{itemize}
	\item \emph{Origins:} Algorithms that try to mimic brain.
	\item Widely used in 80s and early 90s; diminished in late 90s.
	\item Resurgance: State-of-the-art technique for many applications.
\end{itemize}

\section{Model Representation}
\subsection*{Neuron Model: Logistic Unit}
\begin{enumerate}
	\item input layer
	\item hidden layer / computation layer
	\item output layer
	\item activation function ($g()$)
\end{enumerate}
parameters $\equiv$ weights

\section{Notations}
\begin{align*}
	a_i^{(j)}    = & \text{ \emph{activation} of unit } i \text{ in layer } j \\
	\Theta^{(j)} = & \text{ matrix of weights controlling}                    \\
	               & \text{ function mapping from layer } j \text{ to}        \\
	               & \text{ layer } j + 1                                     \\
\end{align*}

\paragraph{Example:}
\begin{align*}
	\text{layer 1}          & : \{x_1, x_2, x_3\}                   \\
	\text{layer 2}          & : \{a_1^{(2)}, a_2^{(2)}, a_3^{(2)}\} \\
	\text{layer 3}          & : \{a_1^{(3)}\}                       \\
	\\
	a_1^{(2)}               & = g\left(
	\Theta_{10}^{(1)}x_0 + \Theta_{11}^{(1)}x_1 + \Theta_{12}^{(1)}x_2 + \Theta_{13}^{(1)}x_3
	\right)                                                         \\
	a_2^{(2)}               & = g\left(
	\Theta_{20}^{(1)}x_0 + \Theta_{21}^{(1)}x_1 + \Theta_{22}^{(1)}x_2 + \Theta_{23}^{(1)}x_3
	\right)                                                         \\
	a_3^{(2)}               & = g\left(
	\Theta_{30}^{(1)}x_0 + \Theta_{31}^{(1)}x_1 + \Theta_{32}^{(1)}x_2 + \Theta_{33}^{(1)}x_3
	\right)                                                         \\
	h_\Theta(x) = a_1^{(3)} & = g\left(
	\Theta_{10}^{(2)}a_0^{(2)} + \Theta_{11}^{(2)}a_1^{(2)} +
	\Theta_{12}^{(2)}a_2^{(2)} + \Theta_{13}^{(1)}a_3^{(2)}
	\right)                                                         \\
\end{align*}

\paragraph{Note:} If network has $s_j$ units in layer $j$, and $s_{j++1}$ units in
layer $j + 1$, then
\Large
$$
	\Theta^{(j)} \in \R^{s_{j+1} \times (s_j + 1)}
$$
\normalsize

\section{Forward Propagation}
\begin{align*}
	z_1^{(2)} & =
	\Theta_{10}^{(1)}x_0 + \Theta_{11}^{(1)}x_1 +
	\Theta_{12}^{(1)}x_2 + \Theta_{13}^{(1)}x_3 \\
	z_2^{(2)} & =
	\Theta_{20}^{(1)}x_0 + \Theta_{21}^{(1)}x_1 +
	\Theta_{22}^{(1)}x_2 + \Theta_{23}^{(1)}x_3 \\
	z_3^{(2)} & =
	\Theta_{30}^{(1)}x_0 + \Theta_{31}^{(1)}x_1 +
	\Theta_{32}^{(1)}x_2 + \Theta_{33}^{(1)}x_3 \\
	a_1^{(2)} & = g(z_1^{(2)})                  \\
	a_2^{(2)} & = g(z_2^{(2)})                  \\
	a_3^{(2)} & = g(z_3^{(2)})                  \\
\end{align*}
\subsection*{Vectorized Implementation}
\begin{align*}
	x       & = \begin{bmatrix}
		x_0  \\
		x_1  \\
		x_2  \\
		x_33 \\
	\end{bmatrix},\ x_0 = 1 \\
	z^{(2)} & = \begin{bmatrix}
		z_1^{(2)} \\
		z_2^{(2)} \\
		z_3^{(2)} \\
	\end{bmatrix}           \\
\end{align*}
\begin{align*}
	a^{(1)}                        & = x
	                               & \in \R^4                     \\
	z^{(2)}                        & = \Theta^{(1)}a^{(1)}
	                               & \in \R^3                     \\
	a^{(2)}                        & = g(z^{(2)})
	                               & \in \R^3                     \\
	\text{\textbf{Add }} a_0^{(2)} & = 1
	                               & \rightarrow a^{(2)} \in \R^4 \\
	z^{(3)}                        & = \Theta^{(2)}a^{(2)}        \\
	h_\Theta(x) = a^{(3)}          & = g(z^{(3)})                 \\
\end{align*}
Other network architechtures are possible!

\chapter{Back Propagation}

\section{Notations}
\begin{align*}
	L   & = \text{total no. of layers in the network}                \\
	s_l & = \text{no. of units (not counting bias unit) in layer } l \\
\end{align*}

\subsection*{Binary Classification}
\begin{equation*}
	y = 0 \text{ or } 1
\end{equation*}
1 output unit
\begin{align*}
	h_\Theta(x) & \in \R                               \\
	s_L         & = 1                                  \\
	K           & = 1    & \text{(for simplification)} \\
\end{align*}

\subsection*{Multi-class Classification (K classes)}
\begin{equation*}
	y \in \R^K
\end{equation*}
\paragraph{Example:} $K = 4$
\begin{equation*}
	y \in \left\{
	\begin{bmatrix}
		1 \\
		0 \\
		0 \\
		0 \\
	\end{bmatrix}, \begin{bmatrix}
		0 \\
		1 \\
		0 \\
		0 \\
	\end{bmatrix}, \begin{bmatrix}
		0 \\
		0 \\
		1 \\
		0 \\
	\end{bmatrix}, \begin{bmatrix}
		0 \\
		0 \\
		0 \\
		1 \\
	\end{bmatrix}
	\right\}
\end{equation*}
$K$ output units
\begin{align*}
	h_\Theta(x) & \in \R^K \\
	s_L         & = K      \\
	K           & \ge 3    \\
\end{align*}


\section{Cost Function}
\subsection*{Logistic Regression}
\begin{equation*}
	J(\theta) = -\left[
		\frac{1}{m}\sum_{i=1}^m\left(
		y^{(i)}\log h_\theta(x^{(i)}) +
		(1 - y^{(i)})\log(1 - h_\theta(x^{(i)})
		\right)
		\right] + \frac{\lambda}{2m}\sum_{j=1}^n\theta_j^2
\end{equation*}
\subsection*{Neural Network}
\begin{equation*}
	h_\Theta(x) \in \R^K \ \ \ \ (h_\Theta(x))_i = i^{th} \text{output}
\end{equation*}
\begin{align*}
	J(\Theta) & = -\frac{1}{m}\left[
	\sum_{i=1}^m \sum_{k=1}^K \left(
	y_k^{(i)}\log\left(
	h_\Theta(x^{(i)})
	\right)_k + \left(
	1 - y_k^{(i)}
	\right)\log\left(
	1 - \left(
		h_\Theta(x^{(i)})
		\right)_k
	\right)
	\right)
	\right]                          \\
	          & + \frac{\lambda}{2m}
	\sum_{l=1}^{L-1}\sum_{i=1}^{s_l}\sum_{j=1}^{s_{l+1}}\left(
	\Theta_{ji}^{(l)}
	\right)^2                        \\
\end{align*}

\section{Backpropagation Algorithm}
\subsection{Gradient Computation}
\begin{align*}
	J(\Theta) & = -\frac{1}{m}\left[
	\sum_{i=1}^m \sum_{k=1}^K \left(
	y_k^{(i)}\log\left(
	h_\Theta(x^{(i)})
	\right)_k + \left(
	1 - y_k^{(i)}
	\right)\log\left(
	1 - \left(
		h_\Theta(x^{(i)})
		\right)_k
	\right)
	\right)
	\right]                          \\
	          & + \frac{\lambda}{2m}
	\sum_{l=1}^{L-1}\sum_{i=1}^{s_l}\sum_{j=1}^{s_{l+1}}\left(
	\Theta_{ji}^{(l)}
	\right)^2                        \\
	\min_\Theta J(\Theta)            \\
\end{align*}

\paragraph{Need code to compute:}
\begin{itemize}
	\item $J(\Theta)$
	\item $\frac{\partial}{\partial\Theta_{ij}^{(l)}}J(\Theta)$
\end{itemize}

\paragraph{Given one training example} $(x, y)$:

Forward Propagation:
\begin{align*}
	a^{(1)}               & = x                                            \\
	z^{(2)}               & = \Theta^{(1)}a^{(1)}                          \\
	a^{(2)}               & = g(z^{(2)})          & (\text{add }a_0^{(2)}) \\
	z^{(3)}               & = \Theta^{(2)}a^{(2)}                          \\
	a^{(3)}               & = g(z^{(3)})          & (\text{add }a_0^{(3)}) \\
	z^{(4)}               & = \Theta^{(3)}a^{(3)}                          \\
	h_\Theta(x) = a^{(4)} & = g(z^{(4)})                                   \\
\end{align*}

\paragraph{Gradient Computation:} Back Propagation

Intuition: $\delta_j^{(l)} = $ ``error'' of node $j$ in layer $l$

For each output unit ($L = 4$)
\begin{align*}
	\delta_j^{(4)} & = a_j^{(4)} - y_j \\
\end{align*}
Vectorize:
\begin{align*}
	\delta^{(4)} & = a^{(4)} - y                               \\
	\delta^{(3)} & = (\Theta^{(3)})^T\delta^{(4)}.*g'(z^{(3)}) \\
	g'(z^{(3)})  & = a^{(3)} .* (1 - a^{(3)})                  \\
	\delta^{(2)} & = (\Theta^{(2)})^T\delta^{(3)}.*g'(z^{(2)}) \\
	g'(z^{(2)})  & = a^{(2)} .* (1 - a^{(2)})                  \\
\end{align*}
No $\delta^{(1)}$
\begin{equation*}
	\frac{\partial}{\partial\Theta_{ij}^{(l)}} J(\Theta) =
	a_j^{(l)}\delta_i^{(l+1)}\ \ \ \ (\text{ignoring $\lambda$; if $\lambda = 0$})
\end{equation*}

\subsubsection*{Backpropagation Algorithm}
\begin{itemize}[label={}]
	\item Training set $\displaystyle \{(x^{(1)}, y^{(1)}), \dots, (x^{(m)}, y^{(m)})\}$
	\item Set $\displaystyle \Delta_{ij}^{(l)} = 0$ (for all $l, i, j$)
	      $\rightarrow \text{ accumulators to compute }
		      \frac{\partial}{\partial\Theta_{ij}^{(l)}J(\Theta)}$
	\item For $i = 1$ to $m$
	      \begin{itemize}[label={}]
		      \item Set $a^{(1)} = x^{(1)}$
		      \item Perform forward propagation to compute $a^{(l)}$ for $l = 2, 3, \dots, L$
		      \item Using $y^{(i)}$, compute $\delta^{(L)} = a^{(L)} - y^{(i)}$
		      \item Compute $\delta^{(L-1)}, \delta^{(L-2)}, \dots, \delta^{(2)}$
		      \item $\Delta_{ij}^{(l)} := \Delta_{ij}^{(l)} + a_j^{(l)}\delta_i^{(l+1)}$
		      \item $\Delta^{(l)} := \Delta^{(l)} + \delta^{(l+1)}(a^{(l)})^T$ (Vectorized)
	      \end{itemize}
	\item $D_{ij}^{(l)} := \begin{cases}
			      \frac{1}{m}\Delta_{ij}^{(l)} +
			      \lambda \Theta_{ij}^{(l)}    & \text{ if } j \neq 0 \\
			      \\
			      \frac{1}{m}\Delta_{ij}^{(l)} & \text{ if } j = 0    \\
		      \end{cases}$
	\item $\frac{\partial}{\partial\Theta_{ij}^{(l)}}J(\Theta) = D_{ij}^{(l)}$
\end{itemize}

\section{Implementation Note}
\subsection{Unrolling Parameters}
Advanced optimization functions like `fminunc' assume that inputs `theta', `gradient', etc. are
vectors. These are matrices in Neural Networks, so we \emph{unroll} them into vectors.

\paragraph{Example:}
Neural Network with 3 layers (1 input, 1 hidden, 1 output) and
$s_1 = 10$, $s_2 = 10$, $s_3 = 1$.
\begin{align*}
	\Theta^{(1)} \in \R^{10 \times 11},\ \Theta^{(2)} \in \R^{10 \times 11},\
	\Theta^{(3)} \in \R^{(1 \times 11)} \\
	D^{(1)} \in \R^{10 \times 11},\ D^{(2)} \in \R^{10 \times 11},\
	D^{(3)} \in \R^{(1 \times 11)}      \\
\end{align*}

\paragraph{To unroll:}
\begin{minted}{octave}
	thetaVec = [Theta1(:); Theta2(:); Theta3(:)];
	DVec = [D1(:); D2(:); D3(:)];
\end{minted}

\paragraph{To go back to matrices:}
\begin{minted}{octave}
	Theta1 = reshape(thetaVec(1:110), 10, 11);
	Theta2 = reshape(thetaVec(111:220), 10, 11);
	Theta3 = reshape(thetaVec(221:231), 1, 11);
\end{minted}

\subsection{Learning Algorithm}
\begin{itemize}[label={}]
	\item Have initial parameters $\Theta^{(1)}, \Theta^{(2)}, \Theta^{(3)}$
	\item Unroll to get \mintinline{octave}{initialTheta} to pass to
	\item \mintinline{octave}{fminunc(@costFunction, initialTheta, options)}
	\item \mintinline{octave}{function [jVal, gradientVec] = costFunction(thetaVec)}
	      \begin{itemize}[label={}]
		      \item From \mintinline{octave}{thetaVec},
		            get $\Theta^{(1)}$, $\Theta^{(2)}$, $\Theta^{(3)}$
		            using \mintinline{octave}{reshape}
		      \item Use forward/back propagation to compute
		            $D^{(1)}$, $D^{(2)}$, $D^{(3)}$ and $J(\Theta)$
		      \item Unroll $D^{(1)}$, $D^{(2)}$, $D^{(3)}$ to get
		            \mintinline{octave}{gradientVec}
	      \end{itemize}
\end{itemize}

\section{Gradient Checking}
An unfortunate problem is that it might seem like it's working even when there's a
bug in the backpropagation algorithm (implementation). Subtle bugs can hence cause
higher level of error and perform worse than a bug-free implementation. Here, we
try to make sure that our implementation is 100\% correct.

\subsection{Numerical Estimation of gradients}
To estimate derivative of $J(\theta)$, we calculate $J(\theta + \epsilon)$
and $J(\theta - \epsilon)$ and take the slope of the line connecting these two points.
\begin{equation*}
	J'(\theta) \approx \frac{J(\theta + \epsilon) - J(\theta - \epsilon)}{2\epsilon}
\end{equation*}
Use pretty small $\epsilon\ (\approx 10^{-4})$.

The above formula is called the
two sided difference. Another formula called the one sided difference is
\begin{equation*}
	J'(\theta) \approx \frac{J(\theta + \epsilon) - J(\theta)}{\epsilon}
\end{equation*}
But the two sided difference usually gives a slightly more accurate approximation.

\paragraph{Implement:}
\begin{minted}{octave}
	gradApprox = (J(theta + EPSILON) - J(theta - EPSILON)) / (2*EPSILON)
\end{minted}

\subsubsection*{Parameter \emph{vector} $\theta$}
\begin{align*}
	\theta                                     & \in \R^n                     \\
	\theta                                     & = \begin{bmatrix}
		\theta_1 \\
		\theta_2 \\
		\vdots   \\
		\theta_n \\
	\end{bmatrix} \\
	\frac{\partial}{\partial\theta_1}J(\theta) & \approx
	\frac{J(\theta_1 + \epsilon, \theta_2, \theta_3, \dots, \theta_n) -
		J(\theta_1 - \epsilon, \theta_2, \theta_3, \dots, \theta_n)}{2\epsilon}   \\
	\frac{\partial}{\partial\theta_2}J(\theta) & \approx
	\frac{J(\theta_1, \theta_2 + \epsilon, \theta_3, \dots, \theta_n) -
		J(\theta_1, \theta_2 - \epsilon, \theta_3, \dots, \theta_n)}{2\epsilon}   \\
	                                           & \vdots                       \\
	\frac{\partial}{\partial\theta_n}J(\theta) & \approx
	\frac{J(\theta_1, \theta_2, \theta_3, \dots, \theta_n + \epsilon) -
		J(\theta_1, \theta_2, \theta_3, \dots, \theta_n - \epsilon)}{2\epsilon}   \\
\end{align*}
\paragraph{Implement:}
\begin{minted}{octave}
	for i = 1:n
		thetaPlus = theta;
		thetaPlus(i) = thetaPlus(i) + EPSILON;
		thetaMinus = theta;
		thetaMinus(i) = thetaMinus(i) - EPSILON;
		gradApprox(i) = (J(thetaPlus) - J(thetaMinus))/(2*EPSILON);
	end;
\end{minted}

Check that \mintinline{octave}{gradApprox} $\approx$ \mintinline{octave}{DVec} (from
backpropagation).

\subsection{Implementation Note:}
\begin{itemize}
	\item Implement backpropagation to compute \mintinline{octave}{Dvec} (unrolled
	      $D^{(1)}$, $D^{(2)}$, $\dots$)
	\item Implement numerical gradient check to compute \mintinline{octave}{gradApprox}
	\item Make sure they give similar values
	\item Turn off gradient checking (much slower) and use backpropagation for learning
	      (much faster)
\end{itemize}

\paragraph{Important:}
\begin{itemize}
	\item Be sure to disable your gradient checking code before training your
	      classifier. If you run numerical gradient computaion on every iteration of
	      gradient descent (or in the inner loop of \mintinline{octave}{costFunction(...)})
	      your code will be \uline{very} slow.
\end{itemize}

\section{Random Initialization}
\subsection*{Initial value of $\Theta$}
For gradient descent and advanced optimization methods, we need initial value of
$\Theta$.

Consider gradient descent:\\
Set \mintinline{octave}{initialTheta = zeroes(n, 1)} ?

Works fine for linear/logistic regression, but not when we're training a
neural network. Why?
\begin{align*}
	\Theta_{ij}^{(l)}                    & = 0 \text{ for all } i, j, l \\
	a_1^{(2)}                            & = a_2^{(2)}                  \\
	\delta_1^{(2)}                       & = \delta_2^{(2)}             \\
	\frac{\partial}
	{\partial\Theta_{01}^{(1)}}J(\Theta) & = \frac{\partial}
	{\partial\Theta_{02}^{(1)}}J(\Theta)                                \\
\end{align*}
Meaning, after each gradient descent update, parameters corresponding to inputs
going into each unit of the next layer are identical. Thus, all the units of the next
layer are still computing the same function of the input. The values may change, but
will always be the same, preventing the neural network from learning something interesting
since it is extremely redundant.

\subsubsection{Random Initialization: Symmetry Breaking}
To get around this (\emph{symmetric weights}), we do random initialization.
Initialize each $\Theta_{ij}^{(l)}$ to a random value in $[-\epsilon, \epsilon]$
(i.e. $-epsilon \leq \Theta_{ij}^{(l)} \leq \epsilon$).

\paragraph{Example:}
\begin{minted}{octave}
	Theta1 = rand(10,11)*(2*INIT_EPSILON) - INIT_EPSILON
	Theta2 = rand(1, 11)*(2*INIT_EPSILON) - INIT_EPSILON
\end{minted}

\section{Putting it all together}
% \subsection{Training a neural network}
Pick a network architecture (connectivity pattern between neurons), like number of
layers, number of units in each layer, etc. For example, $[3 \rightarrow 5 \rightarrow 4]$
vs $[3 \rightarrow 5 \rightarrow 5 \rightarrow 4]$ vs $[3 \rightarrow 5 \rightarrow 5
			\rightarrow 5 \rightarrow 4]$.
\begin{align*}
	\text{No. of input units}  & : \text{Dimension of features } x^{(i)} \\
	\text{No. of output units} & : \text{Number of classes}              \\
\end{align*}
\paragraph{Remember:}
\begin{align*}
	y & \in \{1, 2, \dots, 10\}	\text{ becomes } \\
	y & \in \left\{
	\begin{bmatrix}
		1      \\
		0      \\
		\vdots \\
		0      \\
	\end{bmatrix},
	\begin{bmatrix}
		0      \\
		1      \\
		\vdots \\
		0      \\
	\end{bmatrix}, \dots,
	\begin{bmatrix}
		0      \\
		0      \\
		\vdots \\
		1      \\
	\end{bmatrix}
	\right\}
\end{align*}

\subsubsection{For Hidden Layers?}
\paragraph{Reasonable Default:} $1$ hidden layer, or if $>1$ hidden layer, have same
no. of hidden units in every layer (usually, the more the better, just that it may
get computationally expensive). Usually, comparable to the number of features.

\subsubsection{Training a neural network:}
\begin{enumerate}
	\item Randomly initialize weights
	\item Implement forward propagation to get $h_\Theta(x)$ for any $x$
	\item Implement code to compute cost function $J(\Theta)$
	\item Implement backpropagation to compute partial derivatives
	      $\frac{\partial}{\partial\Theta_{ij}^{(l)}J(\Theta)}$
	      \begin{minted}[escapeinside=||, mathescape=true]{octave}
	for i = 1:m
		Perform forward propagation ... 
			and backpropagation ... 
			using example |$(x^{(i)}, y^{(i)})$|
				
		Get activations |$a^{(l)}$| ... 
			and delta terms |$\delta^{(l)}$| ... 
			for |$l = 2, \dots, L$|

		|$\Delta^{(l)} := \Delta^{(l)} + \delta^{(l+1)}(a^{(l)})^T$|
		|$\vdots$|
	end
	|$\vdots$|
	compute |$\frac{\partial}{\partial\Theta_{jk}^{(l)}}J(\Theta)$|
\end{minted}
	\item \begin{enumerate}
		      \item Use gradient checking to compare $J'(\Theta)$ computed using
		            backpropagation vs using numerical estimate.
		      \item Then disable gradient checking code.
	      \end{enumerate}
	\item Use gradient descent or advanced optimization method with backpropagation
	      to try to minimize $J(\Theta)$
\end{enumerate}

$J(\Theta)$ in general for a neural network is non-convex, and hence susceptible to
local optima. However, it turns out, in practice, this is not usually a huge problem.

\blfootnote{Check \href{lecture_pdf/Lecture9.pdf}{Lecture9.pdf} for more details.}

\chapter{Applying Machine Learning}
\section{Deciding what to try next}
\subsection{Debugging a Learning Algorithm}
Suppose you have implemented regularized linear regression to predict housing prices.
\begin{equation*}
	J(\theta)  = \frac{1}{2m}\left[
		\sum_{i=1}^m\left(
		h_\theta(x^{(i)}) - y^{(i)}
		\right)^2 + \lambda\sum_{j=1}^n\theta_j^2
		\right]
\end{equation*}
However, it makes \emph{unacceptably} large errors in prediction. What to try next?
\begin{itemize}
	\item Get more training examples (Sometimes doesn't actually help)
	\item Try smaller set of features (Overfitting?)
	\item Try additional features (Underfitting?) (Huge project)
	\item Try adding polynomial features
	\item Increasing/Decreasing $\lambda$
\end{itemize}
Need to understand what would help.

\subsection{Machine Learning Diagnostics}
\paragraph{Diagnostic:} A test that you can run to gain insight at what is/isn't working
with the algorithm, and gain guidance as to how best to improve its performance.

Diagnostics can take time to implement, but doing so can be a very good use of your
time as it can save huge time later.

\section{Evaluating your Hypothesis}
Just \emph{training error} not a good indicator (Overfitting).

Solution? Plot Hypothesis function? Not feasible with high number of features.

\subsection{The Standard Way}
Split the dataset into two portions - \emph{Training Set} and \emph{Test Set} with
about $70\%$-$30\%$ ratio (\emph{randomly} if it is in some order).

\paragraph{Notation:}
\begin{align*}
	\text{Training Examples}        & : (x, y)               \\
	\text{Test Examples}            & : (x_{test}, y_{test}) \\
	\text{No. of Training Examples} & : m                    \\
	\text{No. of Test Examples}     & : m_{test}             \\
\end{align*}

\subsection{Train/Test Procedure}
\begin{enumerate}
	\item Learn $\theta$ from training data (minimize $J(\theta)$)
	\item Compute test set error ($J_{test}(\theta)$) on test examples
	      \begin{itemize}
		      \item In case of classification problem, we can also use misclassification
		            error (Also called $0/1$ misclassification error)
		      \item $\text{err}(h_\theta(x), y) = \begin{cases}
				            1 & \text{ if } h_\theta(x) \ge 0.5,\ y == 0   \\
				              & \text{ or if } h_\theta(x) < 0.5, \ y == 1 \\
				            0 & \text{ otherwise }                         \\
			            \end{cases}$
		      \item $\text{test error} = \frac{1}{m_{test}}
			            \sum_{i=1}^{m_{test}}\text{err}
			            (h_\theta(x_{test}^{(i)}), y_{test}^{(i)})$
		      \item This gives us the proportion of the test data that was misclassified
	      \end{itemize}
\end{enumerate}

\section{Model Selection}
Once parameters $\theta$ were fit to some training set, the error $J(\theta)$ measured
on that data is likely to be lower than the actual generalization error.

\paragraph{Model Selection:} Let's say we try to choose what degree polynomial
to fit to data (linear, quadradic, cubic, \dots)

\paragraph{Notation:}
\begin{align*}
	d & = \text{Degree of polynomial} \\
\end{align*}

For different $d$, we can have the following:
\begin{enumerate}
	\item $h_\theta(x) = \theta_0 + \theta_1x$
	\item $h_\theta(x) = \theta_0 + \theta_1x + \theta_2x^2$
	\item $h_\theta(x) = \theta_0 + \theta_1x + \dots + \theta_3x^3$\\
	      \vdots
	\item[10.] $h_\theta(x) = \theta_0 + \theta_1x + \dots + \theta_{10}x^{10}$
\end{enumerate}

Choose one model out of the above. We can train them separately, and get different
$\theta^{(k)}$, and calculate $J_{test}(\theta)$ for each of them, and choose
the one with the least test error.

How well does this model generalize? Report $J_{test}(\theta)$.

\paragraph{Problem:} $J_{test}(\theta)$ is likely an optimistic estimate of
generalization error (our extra parameter $d$ is fit to \emph{test set}).

\subsection{Evaluating your hypothesis}
Divide the data set in three pieces, \emph{training set} ($60\%$), \emph{cross-validation
	set} ($20\%$), \emph{test set} ($20\%$).

\paragraph{Notations:}
\begin{align*}
	\text{Training Examples}                & : (x, y)                                  \\
	\text{Cross-Validation Examples}        & : (x_{cv}, y_{cv})                        \\
	\text{Test Examples}                    & : (x_{test}, y_{test})                    \\
	\text{No. of Training Examples}         & : m                                       \\
	\text{No. of Cross-Validation Examples} & : m_{cv}                                  \\
	\text{No. of Test Examples}             & : m_{test}                                \\
	\text{Training Error}                   & : J(\theta) \text{ or } J_{train}(\theta) \\
	\text{Cross Validation Error}           & : J_{cv}(\theta)                          \\
	\text{Test Error}                       & : J_{test}(\theta)                        \\
\end{align*}

Use \emph{Cross-Validation Set} to select the model. Estimate generalization
error for test set.

\section{Bias vs Variance}
Plot \emph{error} against \emph{degree} of the polynomial for both, \emph{training}
and \emph{validation} data set. Usually, training error tends to decrease. Cross-Validation
error (or test error) on the other hand, tends to decrease, and then increase again.

To recognise bias/variance problem:
\begin{align*}
	\text{Bias (Underfit)}    & : & J_{train} \text{ will be high} \\
	                          &   & J_{cv} \approx J_{train}       \\
	\text{Variance (Overfit)} & : & J_{train} \text{ will be low}  \\
	                          &   & J_{cv} >> J_{train}            \\
\end{align*}

\subsection{Effect of regularization}
Large $\lambda$ can cause high bias (underfit), and at very small $\lambda$ causes
high variance (overfit). How to automatically choose a good value of $\lambda$.

\paragraph{Choosing $\lambda$:}
\begin{align*}
	J(\theta)         & = \frac{1}{2m}\sum_{i=1}^m(h_\theta(x^{(i)}) - y^{(i)}) +
	\frac{\lambda}{2m}\sum_{j=1}^m\theta_j^2                                      \\
	J_{train}(\theta) & = \frac{1}{2m}\sum_{i=1}^m(h_\theta(x^{(i)}) - y^{(i)})   \\
	J_{cv}(\theta)    & = \frac{1}{2m_{cv}}
	\sum_{i=1}^{m_{cv}}(h_\theta(x_{cv}^{(i)}) - y_{cv}^{(i)})                    \\
	J_{test}(\theta)  & = \frac{1}{2m_{test}}
	\sum_{i=1}^{m_{test}}(h_\theta(x_{test}^{(i)}) - y_{test}^{(i)})              \\
\end{align*}

Have a model ($h_\theta(x)$, $J(\theta)$), have a bunch of values to try for $\lambda$ like $0$,
$0.01$, $0.02$, $0.04$, $0.08$, \dots, $10$ (roughly doubling). Train (minimize $J(\theta)$)
for all values. Use $J_{cv}(\theta)$ to choose a $\lambda$. Report generalization error as
$J_{test}(\theta)$.

\subsubsection{Bias/Variance as a function of $\lambda$}
Plot $J_{train}(\theta)$ and $J_{cv}(\theta)$ as a function of $\lambda$. Usually,
$J_{train}(\theta)$ tends to increase while $J_{cv}(\theta)$ tends to decrease at
first but increase again later.

\section{Learning Curves}
Useful thing to plot, to sanity check, to improve the performance, to diagnose for
bias/variance, \dots.

Plot $J_{train}(\theta)$, and $J_{cv}(\theta)$ as a function of $m$, the number of
training examples. Measure $J_{train}(\theta)$ only on data used to train!. Measure
$J_{cv}(\theta)$ on all cross validation data!

Usually, average training error will increase with $m$.
Usually, average validation error will decrease with $m$.

\subsection{High Bias}
\paragraph{Usually:}
Validation error will decrease at first but then will saturate/flatten out soon.
Training error will increase at first but then will saturate/flatten out soon, close to
validation error.

\paragraph{Note:} If a learning algorithm is suffering from high bias, getting more training
data will not (by itself) help much.

\subsection{High Variance}
\paragraph{Usually:}
Training error will increase but still be pretty low.
Validation error will decrease but still be pretty high, far from training error.

\paragraph{Note:} If a learning algorithm is suffering from high variance, getting more
training examples is indeed likely to help.

\section{Deciding what to try next (Revisited)}
\begin{tabular}{| l | l |}
	\hline
	Get more training examples      & fixes high variance          \\
	\hline
	Try smaller set of features     & fixes high variance          \\
	\hline
	Try additional features         & fixes high bias (not always) \\
	\hline
	Try adding polynomial features  & fixes high bias (not always) \\
	\hline
	Increasing/Decreasing $\lambda$ & fixes bias/variance          \\
	\hline
\end{tabular}

\subsection{Neural Networks}
\emph{Small} neural network, computationally cheaper, but possibly underfitting.
\emph{Large} neural networks, computationally more expensive, and possibly overfitting.
Use regularization to address overfitting.

Often, large neural network with regularization works better (but may be computationally
expensive).

Large network can comprise of a single large hidden layer, or two moderate hidden layers, or
three average hidden layers, etc.

\blfootnote{Check \href{lecture_pdf/Lecture10.pdf}{Lecture10.pdf} for more details.}

%%% chapter endnote
% \blfootnote{Check \href{lecture_pdf/Lecture1.pdf}{Lecture1.pdf} for more details.}

% \begin{appendices}
% \end{appendices}

\end{document}