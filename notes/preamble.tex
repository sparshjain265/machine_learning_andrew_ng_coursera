\documentclass[a4paper, 12pt]{report}
\usepackage{graphicx}
\usepackage{amsmath, amsthm, amssymb}
\usepackage{fourier}
\usepackage{enumerate}
\usepackage{enumitem}
\usepackage{array}
\usepackage{tabularx}
\usepackage{hyperref}
\usepackage{caption}
\usepackage[normalem]{ulem}
\usepackage{pdfpages}
\usepackage[toc,page]{appendix}
\usepackage{float}

\usepackage[chapter,newfloat]{minted}
\SetupFloatingEnvironment{listing}{name=Program}
\SetupFloatingEnvironment{listing}{listname=List of Programs}
\newenvironment{code}{\captionsetup{type=listing}}{}

%%% blank footnote
\newcommand\blfootnote[1]{
    \begingroup
    \renewcommand\thefootnote{}\footnote{#1}
    \addtocounter{footnote}{-1}
    \endgroup
}
%%%

%%% prevent hyphenation
\tolerance=1
\emergencystretch=\maxdimen
\hyphenpenalty=10000
\hbadness=10000
%%%

%%% make inline math look like display style math
\everymath{\displaystyle}

%%% minimize
\DeclareMathOperator*{\minimize}{minimize}
%%%

% Theorem environments
\newtheorem{theorem}{Theorem}
\newtheorem{corollary}[theorem]{Corollary}
\newtheorem{lemma}[theorem]{Lemma}
\newtheorem{proposition}[theorem]{Proposition}
\newtheorem{conjecture}[theorem]{Conjecture}
\newtheorem{observation}[theorem]{Observation}

\theoremstyle{definition}
\newtheorem{definition}[theorem]{Definition}
\newtheorem{question}[theorem]{Question}

\theoremstyle{remark}
\newtheorem*{remark}{Remark}

%%%
\def\N{\mathbb{N}}
\def\Z{\mathbb{Z}}
\def\Q{\mathbb{Q}}
\def\R{\mathbb{R}}
\def\F{\mathbb{F}}

\def\tends{\rightarrow}
\def\into{\rightarrow}
\def\half{\frac{1}{2}}
\def\quarter{\frac{1}{4}}

\newcommand{\set}[1]{\left\{ #1 \right\}}
\newcommand{\norm}[1]{\left\Vert #1 \right\Vert}
\newcommand{\card}[1]{\left\vert #1 \right\vert}
\newcommand{\ceil}[1]{\left\lceil #1 \right\rceil}
\newcommand{\floor}[1]{\left\lfloor #1 \right\rfloor}

\DeclareMathOperator{\dia}{dia}
%%%

\title{Notes\\
    \large Machine Learning by Andrew Ng on Coursera}
\author{Sparsh Jain}
\date{\today}