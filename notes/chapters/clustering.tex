\chapter{Clustering}
\section{Unsupervised Learning}
Learning from unlabelled data as opposed to supervised learning. Give data to the algorithm
to find some \emph{structure} in our data.

\section{Clustering}
Group data in different \emph{clusters}.

\subsection{Application}
\begin{itemize}
	\item Market Segmentation
	\item Social Network Analysis
	\item Organize Computer Clusters
	\item Astronomical Data Analysis
\end{itemize}

\section{$K$-means}
\subsection{Algorithm}
\paragraph{Informally:}
Suppose we want to group the data into two clusters.
\begin{enumerate}
	\item Randomly initialize two cluster centroids.
	\item Assign each data point to a cluster centroid based on some \emph{distance}
	      measure.
	\item Update cluster centroid to the average of each data point in the same
	      cluster.
	\item Go back to step 2 until convergence.
\end{enumerate}

\paragraph{Formally}:\\
Input:
\begin{itemize}
	\item $K$ (number of clusters)
	\item Training set $\{x^{(1)}, x^{(2)}, \dots, x^{(m)}\}$
\end{itemize}

\begin{remark}
	$x^{(i)} \in \R^n$ (Drop $x_0 = 1$ as convention)
\end{remark}

\begin{minted}[escapeinside=||]{c}
Randomly initialize |$K$| cluster centroids |$\mu_1, \mu_2, \dots, \mu_K \in \R^n$|
Repeat {
	for |$i$| = 1 to |$m$|
		|$c^{(i)}$| := index (from 1 to |$K$|) of cluster centroid
			closest to |$x^{(i)}$|
	for |$k$| = 1 to |$K$|
		|$\mu_k$| := average (mean) of points assigned to cluster |$k$|
}
\end{minted}

\begin{remark}
	If we get a cluster centroid with zero data points in it, the common thing to do is to
	entirely eliminate that centroid and get $K-1$ clusters. But if you really need $K$
	clusters, you can randomly reinitialize the centroid.
\end{remark}

\subsection{$K$-means for non-separated clusters}
\paragraph{Example:} T-shirt sizing (S/M/L) based on weight and height.

\subsection{Optimization Objective}
\subsubsection{Notations}
\begin{align*}
	c^{(i)}       & = \text{index of cluster } (1, 2, \dots, K) \text{ to which example }
	x^{(i)} \text{ is currently}                                                          \\
	              & \ \ \ \ \text{assigned}                                               \\
	\mu_k         & = \text{cluster centroid } k\  (\mu_k \in \R^n)                       \\
	\mu_{c^{(i)}} & = \text{cluster centroid of cluster to which example } x^{(i)}
	\text{ has been }                                                                     \\
	              & \ \ \ \ \text{assigned}                                               \\
\end{align*}

\subsubsection{Cost Function}
Also called \emph{distortion} cost function or
\emph{distortion} of the $K$-means algorithm.
\begin{equation*}
	J(c^{(1)}, \dots, c^{(m)}, \mu_1, \dots, \mu_K) =
	\frac{1}{m} \sum_{i=1}^m \norm{x^{(i)} - \mu_{c^{(i)}}}^2
\end{equation*}

\subsubsection{Optimization Objective}
\begin{equation*}
	\min_{\substack{c^{(1)}, \dots, c^{(m)},\\ \mu_1, \dots, \mu_K}}
	J(c^{(1)}, \dots, c^{(m)}, \mu_1, \dots, \mu_K)
\end{equation*}

\subsection{Random Initialization}
Many ways, but this one usually works better in most cases.
Should have $K < m$. Randomly pick $K$ training examples as
cluster centroids.

\subsubsection{Local Optima}
On bad days, $K$-means can get stuck in local optima. To avoid, we can initialize
multiple times. Concretely, repeat about 100 times (usually 50 to 1000 times) and
choose the way with least distortion.

For small $K$ (2 to 10), multiple initialization is likely to give better solution
while for larger $K$s, it's highly likely that your first solution was pretty good
already and not make a huge difference.

\subsection{Choosing \texorpdfstring{$K$}{}}
Still the most common thing is to choose the number of clusters by hand.
Below are other thoughs.

\subsubsection{Elbow Method}
Plot distortion against the number of clusters. Find the \emph{elbow}
in the plot. Maybe that is a good number of clusters. Pretty good way
if the graph goes down rapidly and then slowly later. But, not used that
often coz often the elbow is very often very ambiguous.

Sometimes, we run $K$-means to get clusters for some later purpose. Evaluate
based on metric for how well it performs for that later purpose.

\blfootnote{Check \href{lecture_pdf/Lecture13.pdf}{Lecture13.pdf} for more details.}