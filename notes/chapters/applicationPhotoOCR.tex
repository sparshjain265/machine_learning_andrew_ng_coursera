\chapter{Application Example: Photo OCR}
\section{Problem Description and Pipeline}
\subsection{Photo Optical Character Recognition} It focuses to read the text
in an image. Very difficult compared to PDF OCR.

\subsection{Photo OCR Pipeline}
\begin{enumerate}
	\item Text detection
	\item Character segmentation
	\item Character classification
	\item Spelling Correction (Optional)
\end{enumerate}
Different people working on different parts of a pipeline!

\section{Sliding Windows}
\subsection{Pedestrial Detection: Supervised Learning}
Aspect Ratio in a \emph{Pedestrian Detection} is roughly fixed while not so
in text detection.
\begin{enumerate}
	\item $x = $ pixels in $82 \times 36$ image patches
	\item Get large training set of positive and negative examples
	\item Learn, given an image patch, whether or not a pedestrian exists in it
	\item \textbf{Sliding Window Detection:} Take a rectangular patch and shift it
	      throughout the image and check for pedestrian. The \emph{step size} or \emph{stride}
	      to slide the \emph{window} works best at $1$ but is more computationally expensive
	      so $4$ or $8$ or some large number of pixels at a time is more common.
	\item Then, take larger patches and repeat.
\end{enumerate}
\subsection{Text Detection}
\begin{enumerate}
	\item Similar to pedestrian detection, get large training set of positive
	      and negative examples and train on it.
	\item Apply it to a new test image i.e. run on all image patches (for simplicity,
	      of the same size in our example).
	\item Take output, and apply \emph{expansion operator}. It takes each of the \emph{positive}
	      region and expands it. Specifically, if a pixel is within a specific distance
	      of a positive pixel, mark it positive as well.
	\item We can now look at \emph{connected components} of positive pixels.
	\item Simply discard boxes of \emph{funny} aspect ratios (for texts, boxes should be
	      generally wider than they are tall).
\end{enumerate}

\subsection{1D Sliding Window for character segmentation}
\begin{enumerate}
	\item Again, use supervised learning algorithm.
	\item Look at an image patch, and decide if there is a split right in the middle
	      of the image patch.
	\item Slide in one dimension, and segment the characters.
\end{enumerate}

\subsection{Character classification}
Classic multi-class classification problem!

\section{Artificial Data Synthesis}
Doesn't apply to every problem, and needs thought, idea, innovation, and insight.
\begin{enumerate}
	\item Create data from scratch
	\item Having a small dataset and creating a larger dataset out of it
\end{enumerate}

\subsection{Character Recognition}
\subsubsection{Creating data from scratch}
Fonts in your system, plus font libraries! Take these characters and paste them
on a random background. After some amount of work (synthesise realistic looking
data) like blurring, affine transformation etc.
\subsubsection{Synthesizing data by introducing distortions}
Introduce artificial distortions/warpings to create new examples.

\begin{remark}
	Again, it takes thought and insight to figure out what are reasonable ways to do
	this.
\end{remark}

\subsubsection{Speech Recognition}
Introduce additional audio distortions in the dataset like different background sounds.

\subsection{Remarks:}
\begin{itemize}
	\item Distortions introduced should be representative of the type of
	      noise/distortions in the test set.
	\item Usually does not help to add purely random/meaningless noise to
	      your data.
	\item A little artistic! Sometimes need to apply and see!
	\item Make sure you have a low bias classifier before expending the
	      effort. (Plot learning curves). E.g. keep increasing the number
	      of features/number of hidden units in neural network.
	\item "How much work would it be to get a $10\times$ as much data as
	      we currently have?" \begin{itemize}
		      \item Artificial Data Synthesis (Scratch or Distortions)
		      \item Collect/label it yourself
		      \item "Crowd source" (E.g. Amazon Mechanical Turk)
	      \end{itemize}
\end{itemize}

\section{Ceiling Analysis: What part of the pipeline to work on next}
What part of the pipeline should you spend the most time trying to improve?
\begin{enumerate}
	\item Have a single number evaluation metric.
	\item For each stage of the pipeline, give it the correct labels and
	      measure the evaluation metric again.
	\item The stages which improve the metric hugely with correct labels,
	      have a high chance that improving that stage is going to improve
	      the performance of the whole system.
\end{enumerate}


\blfootnote{Check \href{lecture_pdf/Lecture18.pdf}{Lecture18.pdf} for more details.}